1. Titelfolie: Beginne mit einer Titelfolie, auf der der Titel deiner Arbeit, dein Name und dein Datum stehen.

2. Einleitung: Die Einleitung sollte den Zweck deiner Arbeit beschreiben, warum es wichtig ist und welche Fragen du in deiner Arbeit beantworten willst. Stelle auch kurz deine Methoden vor.

3. Literaturüberblick: Zeige die wichtigsten Erkenntnisse aus der Literatur, die relevant für deine Arbeit sind. Beschränke dich dabei auf die wichtigsten Studien.

4. Methoden: Erkläre deine Methoden und wie du deine Forschungsfragen beantwortet hast.

5. Ergebnisse: Präsentiere deine Ergebnisse und erkläre, was sie bedeuten und wie sie zur Beantwortung deiner Forschungsfragen beitragen.

6. Diskussion: Diskutiere die Ergebnisse und wie sie mit früheren Forschungen übereinstimmen oder sich von ihnen unterscheiden. Gib auch Empfehlungen für weitere Forschungen und Möglichkeiten zur Anwendung deiner Ergebnisse.

7. Fazit: Zusammenfassung der wichtigsten Ergebnisse und wie sie zur Beantwortung der Forschungsfragen beigetragen haben.

8. Ausblick: Gebe einen kurzen Ausblick auf mögliche zukünftige Forschungsrichtungen oder Anwendungen.

9. Literaturverzeichnis: Liste die Quellen auf, die du in deiner Arbeit verwendet hast.

10. Danksagung: Du kannst optional auch eine Danksagung an Personen oder Organisationen aufnehmen, die zur Durchführung deiner Arbeit beigetragen haben.

Es ist wichtig, dass du die Informationen auf deinen Folien klar und präzise präsentierst und nicht zu viele Details auf einer Folie unterbringst. Nutze Schlagwörter und Bilder, um deine Ergebnisse und Schlussfolgerungen zu veranschaulichen und zu unterstützen. Am besten ist es, wenn du eine logische Struktur auf deinen Folien schaffst und Überschriften verwendest, um den Inhalt jeder Folie zu beschreiben. Beachte auch, dass du während des Kolloquiums mündlich erklären musst, was auf den Folien steht. Eine gut strukturierte Präsentation mit klaren Inhalten und sorgfältig gewählten Abbildungen wird dir helfen, einen klaren Überblick zu geben und dein Publikum zu überzeugen.
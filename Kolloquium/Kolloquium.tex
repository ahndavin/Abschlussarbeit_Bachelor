성공적인 프리젠테이션을 통해 귀하는 귀하가 귀하의 주제를 집중적으로 다루었고 전문가가 되었음을 시험 위원회에 보여줄 수 있습니다.
학사 논문을 다시 반복하는 것이 아닙니다. 오히려 자신의 연구를 생생하게 발표하고 설명할 수 있는 기회입니다.

프레젠테이션은 학사 논문의 구조를 기반으로 해야 합니다 .
학사 논문과 결과를 다시 읽으십시오. 이미 주제를 집중적으로 다뤘더라도 다시 한 번 핵심 내용을 주의 깊게 읽는 것이 좋습니다.


0. Agenda

1. Einleitung
    Zweck diser Arbeit: Bedarf von BigData erhöht sich, Low-Cost-Sensor, UmweltMessen kommt häufig Ausr.
    Warum ist das Thema relevant? Umwelt und Zeitreihen wichtig(Abbildung in der Arbeit)
    Die Arbeit antwortet... 
    Anomalie
    Zeitreihen
    Modellierung der Feinst.

3. Literaturüberblick
    Der Unterschied zwischen meiner und anderen

4. Methoden
    Welche Art von Methode verwendet? iqr, z-score, ...
    Wie die Untersuchung vorgegangen? ???

    
    

5. Evaluation
    Wie der Datensatz erhoben?  DEUS-Projekt
    Welche Metrik für Evaluation?  SMAPE
    
6. Fazit & Ausblick
    Fazit
    Weitere Forschungsansätze und Begrenzungen dieser Untersuchung








기존 코드가 작동하는지 확인하고 개인 노트북을 가져와야 하는지 아니면 USB 스틱만 가져와야 하는지 확인하세요.
자유로운 표현과 열린 몸짓 언어는 성적을 평가하는 중요한 기준입니다. 그러니 항상 청중에게 말을 걸고 눈을 마주치도록 하세요.
프레젠테이션 부분에서 토론으로의 전환은 중요한 포인트입니다. 프레젠테이션의 마지막 문장이 끝난 후 미해결 질문을 지적하여 쉽게 토론을 시작할 수 있습니다.




콜로키움의 일반적인 질문
    이 주제를 선택한 이유는 무엇입니까?
    정확히 이 주제로 학사 논문을 쓰게 된 동기는 무엇입니까?
    그 주제가 당신에게 관심을 갖게 된 이유는 무엇입니까?

    사용하는 방법론을 선택한 이유는 무엇입니까?

    (당신의 작업에서) XY에 대해 더 자세히 설명해 주시겠습니까?
    X라는 용어를 많이 사용하는데 자세히 설명해주실 수 있나요?
    왜 A이론 외에 B이론을 사용하지 않았는가?
    우리는 당신의 작업에서 Y점을 놓치고 있습니다. 그것에 대해 뭐라고 대답할 수 있습니까?
    작업 평가에서 X가 누락된 것으로 나타났습니다. 이 비판에 대해 무엇을 말할 수 있습니까?

    귀하의 결과가 귀하의 분야에 어느 정도 부가가치를 제공합니까?
    귀하의 결과를 유사한 사례에 어느 정도까지 적용할 수 있습니까?
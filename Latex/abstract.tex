\begin{abstract}
    In der heutigen Datenwelt ist das maschinelle Lernen ein wertvolles Werkzeug geworden, um komplexe Probleme zu lösen und Entscheidungen zu treffen. Seine Wichtigkeit wurde durch die Anwendung in vielen Bereichen bereits nachgewiesen. Bei der Umsetzung dieser Lernmodelle spielt Big Data die wichtigste Rolle. Die steigende Menge und Qualität von Big Data tragen zur Verbesserung der Qualität von Prognosemodellen bei und führen somit zu besseren Entscheidungen. In Bezug auf das Datenvolumen sammeln und speichern wir bereits ausreichende Daten, aber in Bezug auf die Datenqualität gibt es noch viel zu verbessern. Methoden wie Labeling, Datenbereicherung, Datenreduktion und Ausreißererkennung können dazu beitragen, die Datenqualität zu verbessern. In dieser Arbeit liegt das besondere Augenmerk auf der Ausreißererkennung in Zeitreihendaten, die durch die Erfassung von Umweltdaten gewonnen werden, wo Ausreißer häufig vorkommen.

    Diese vorliegende Arbeit untersucht die Auswirkungen der Ausreißererkennung auf die Vorhersagegenauigkeit von maschinellen Lernmodellen. Im Gegensatz zu früheren Arbeiten, die sich auf einen Algorithmus oder Algorithmen, die auf einer bestimmten Eigenschaft basieren, beschränkten, untersucht diese Arbeit fünf Algorithmen, die auf verschiedenen Eigenschaften beruhen. Die hier verwendeten Algorithmen sind Interquartile Range, z-Score Filter, K-Means Clustering, Local Outlier Factor und Isolation Forest. Das Ziel dieser Arbeit ist es, ein tieferes Verständnis für die verschiedenen Algorithmen zur Ausreißererkennung zu entwickeln und deren Stärken und Schwächen zu identifizieren. Darüber hinaus kann eine gründliche Analyse der vorhandenen Algorithmen dazu beitragen, eine Empfehlung für den besten Algorithmus in einer bestimmten Anwendungssituation zu geben.

    Die Untersuchung führte zu dem Ergebnis, dass alle Algorithmen mit Ausnahme des LOF-Algorithmus die Vorhersagegenauigkeit in stark verzerrten Datensätzen verbessern können.


    Maschinelles Lernen ist ein wichtiges Werkzeug für komplexe Problemlösungen und Entscheidungen. Die Ausreißerentfernung trägt zur Verbesserung der Prognosemodelle bei. Diese Arbeit untersucht die Auswirkungen der Ausreißererkennung auf die Vorhersagegenauigkeit von maschinellen Lernmodellen und vergleicht fünf Algorithmen (Interquartile Range, z-Score Filter, K-Means Clustering, Local Outlier Factor, Isolation Forest) auf ihre Stärken und Schwächen. Das Ziel ist, ein tieferes Verständnis der vorhandenen Algorithmen zu entwickeln und Empfehlungen für den besten Algorithmus in bestimmten Situationen zu geben.  Die Untersuchung führte zu dem Ergebnis, dass alle Algorithmen mit Ausnahme des LOF-Algorithmus die Vorhersagegenauigkeit in stark verzerrten Datensätzen verbessern können.
\end{abstract}
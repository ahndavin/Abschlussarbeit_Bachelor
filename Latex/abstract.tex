\begin{abstract}
    Klarheit und Kürze: Ein Abstract sollte klar und präzise formuliert sein, um einen schnellen Überblick über die Arbeit zu geben. Es sollte auch kurz gehalten werden, in der Regel zwischen 100-300 Wörtern.

    Zusammenfassung der Arbeit: Ein Abstract sollte eine Zusammenfassung der wichtigsten Ergebnisse und Schlussfolgerungen der Arbeit enthalten.
    연구결과 LOF 알고리즘을 제외한 모든 네가지 알고리즘을 통한 이상치제거가 모델의 성능을 향상 시켰다. 

    Ziel und Methoden: Es sollte beschrieben werden, welches Ziel verfolgt wurde und welche Methoden verwendet wurden, um das Ziel zu erreichen.

    Originalität und Beitrag: Ein Abstract sollte auch aufzeigen, was die Arbeit originell und besonders macht und welchen Beitrag sie zum Gebiet leistet.

    Keywords: Es ist wichtig, passende Keywords zu verwenden, um die Arbeit besser zu indexieren und leichter zu finden.


    목적: 작업의 목적과 목표를 명확하게 기술하십시오. - 어떤 문제가 있는지
    머신러닝의 중요성은 이미 여러 분야에 적용됨으로써 증명되었다.
    머신러닝에 사용되는 빅데이터는 중요하다. 데이터의 양도 중요하지만 질도 중요하다.
    데이터의 질을 높이는 방법에는 Labeling, Datenbereicherung, Datenreduktion oder Ausreißererkennung이 있다.
    이 논문에서는 이상치탐지를 통한 예측모델의 성능 변화를 조사한다.
    이상치탐지를 통해 쏟아지는 빅데이터의 퀄리티를 높여 더 나은 예측모델을 구축하고 결과적으로 더 나은 의사 결정을 내릴 수 있게한다.
    

    독창성: 귀하의 작업이 같은 분야의 이전 연구에 어떻게 기여하고 실제로 어떻게 기여하는지 강조하십시오.- 이 결과로 세상이 어떻게 나아졌는지
    다른 논문에서는 한가지 알고리즘이나 한가지 특성을 기반으로하는 알고리즘들만이 사용되었다.
    이 논문에서는 여러 특성을 기반으로하는 다섯가지 알고리즘을 이용하여 이상치를 탐지하고 각 알고리즘들의 강, 약점을 분석한다. 결과로 해당 알고리즘들을 이용해 이상치를 제거한 데이터세트를 예측모델의 훈련데이터로 사용하여 예측모델의 성능을 비교한다.


    작업 요약: 작업 내용에 대한 간략한 개요를 제공하고 가장 중요한 주장과 결과를 제시합니다.
    방법: 사용된 데이터, 절차 및 분석 방법을 포함하여 작업을 완료하는 데 사용한 방법을 설명하십시오.- 어떤 방법을 사용했는지
    Interquartile Range, z-Score Filter, K-Means Clustering, Local Outlier Factor und Isolation Forest 알고리즘을 사용하여 이상치를 검출했다.

    결과: 결론 및 권장 사항을 포함하여 작업의 주요 결과에 대한 명확한 요약을 제공합니다. - 결과는 어떤지
    LOF 알고리즘을 제외한 모든 다른 알고리즘이 예측모델의 성능을 향상하는데 도움이 되었다. 

    키워드: 작업의 주제와 사용한 방법을 설명하는 주요 키워드 목록을 포함합니다.


    Das Ziel dieser Diplomarbeit ist es, ein tieferes Verständnis für die verschiedenen Methoden zur Ausreißererkennung zu entwickeln und deren Stärken und Schwächen zu identifizieren. Durch eine gründliche Analyse der vorhandenen Technologien wird das Ziel erreicht, eine Empfehlung für die beste Methode zur Anwendung in einer bestimmten Anwendungssituation zu geben.

    Darüber hinaus soll ein Vergleich der verschiedenen Methoden auf einem realen Datensatz durchgeführt werden, um die Effizienz und Genauigkeit der Methoden zu bewerten. Die Ergebnisse dieser Studie werden wichtige Informationen für Forscher und Praktiker bereitstellen, die sich mit der Ausreißererkennung beschäftigen.

\end{abstract}


In der heutigen Datenwelt ist das maschinelle Lernen ein wertvolles Werkzeug geworden, um komplexe Probleme zu lösen und Entscheidungen zu treffen. Durch die Anwendung des maschinellen Lernens in vielen Bereichen hat seine Bedeutung bereits bewiesen.

Bei der Umsetzung dieser Lernmodelle spielt Big Data die wichtigste Rolle. Die steigende Menge und Qualität von Big Data tragen stark zur Verbesserung der Qualität von Vorhersagemodellen bei, was wiederum zu einer besseren Entscheidungsfindung führt. Schon heute sammeln wir unzählige Datenmengen. Jetzt geht es darum, die Qualität dieser Daten zu verbessern. Methoden wie Labeling, Datenbereicherung, Datenreduktion und Ausreißererkennung können dazu beitragen, die Datenqualität zu verbessern. Dieses Papier befasst sich mit der Ausreißererkennung unter ihnen.

Diese Arbeit untersucht die Auswirkungen der Ausreißererkennung auf die Vorhersagegenauigkeit von maschinellen Lernmodellen. Im Gegensatz zu früheren Arbeiten, die nur einen Algorithmus verwendeten, werden in dieser Arbeit fünf verschiedene Algorithmen zur Ausreißererkennung eingesetzt und analysiert. Die Algorithmen, die verwendet werden, sind Interquartile Range, z-Score Filter, K-Means Clustering, Local Outlier Factor und Isolation Forest. Die Stärken und Schwächen jedes Algorithmus werden untersucht, und die Vorhersagegenauigkeit jedes Prognosemodells, das auf dem trainierten Datensatz basiert, wird verglichen.

Als Ergebnis wird gezeigt, dass alle Algorithmen mit Ausnahme des LOF-Algorithmus die Vorhersagegenauigkeit verbessern können. Diese Arbeit belegt, wie wichtig es ist, die Technik der Ausreißererkennung im maschinellen Lernen zu verstehen und anzuwenden, um die Qualität von Big Data zu verbessern und bessere Vorhersagemodelle zu erhalten.





In der aktuellen Datenwelt hat sich maschinelles Lernen als ein unverzichtbares Instrument etabliert, um komplexe Herausforderungen anzugehen und Entscheidungen zu treffen. Seine Wichtigkeit wurde durch seine Anwendung in vielen Bereichen bereits nachgewiesen.

Ein wichtiger Faktor bei der Implementierung von maschinellen Lernmodellen ist Big Data. Die steigende Menge und Qualität von Big Data trägt zur Verbesserung der Vorhersagegenauigkeit von Modellen bei, was wiederum zu besseren Entscheidungen führt.

Jedoch müssen wir uns auch um die Qualität der Daten kümmern, die bereits gesammelt werden. Methoden wie Labeling, Datenbereicherung, Datenreduktion und Ausreißererkennung können dabei helfen, die Datenqualität zu verbessern. In dieser Arbeit konzentrieren wir uns auf die Ausreißererkennung.

Die vorliegende Arbeit untersucht den Effekt der Ausreißererkennung auf die Vorhersagegenauigkeit von maschinellen Lernmodellen. Im Gegensatz zu früheren Arbeiten, die nur einen Algorithmus verwendeten, verwenden wir fünf verschiedene Algorithmen zur Ausreißererkennung und analysieren ihre Leistung. Die verwendeten Algorithmen sind Interquartile Range, z-Score Filter, K-Means Clustering, Local Outlier Factor und Isolation Forest. Wir untersuchen die Stärken und Schwächen jedes Algorithmus und vergleichen die Vorhersagegenauigkeit jedes Modells, das auf dem trainierten Datensatz basiert.

Das Ergebnis zeigt, dass alle Algorithmen - mit Ausnahme des LOF-Algorithmus - die Vorhersagegenauigkeit verbessern können. Diese Arbeit belegt, wie wichtig es ist, die Technik der Ausreißererkennung im maschinellen Lernen zu verstehen und anzuwenden, um die Datenqualität von Big Data zu verbessern und bessere Vorhersagemodelle zu erzielen.


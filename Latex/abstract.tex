\begin{abstract}
    Klarheit und Kürze: Ein Abstract sollte klar und präzise formuliert sein, um einen schnellen Überblick über die Arbeit zu geben. Es sollte auch kurz gehalten werden, in der Regel zwischen 100-300 Wörtern.

    Zusammenfassung der Arbeit: Ein Abstract sollte eine Zusammenfassung der wichtigsten Ergebnisse und Schlussfolgerungen der Arbeit enthalten.
    연구결과 LOF 알고리즘을 제외한 모든 네가지 알고리즘을 통한 이상치제거가 모델의 성능을 향상 시켰다. 

    Ziel und Methoden: Es sollte beschrieben werden, welches Ziel verfolgt wurde und welche Methoden verwendet wurden, um das Ziel zu erreichen.

    Originalität und Beitrag: Ein Abstract sollte auch aufzeigen, was die Arbeit originell und besonders macht und welchen Beitrag sie zum Gebiet leistet.

    Keywords: Es ist wichtig, passende Keywords zu verwenden, um die Arbeit besser zu indexieren und leichter zu finden.

    머신러닝의 중요성은 이미 여러 분야에 적용됨으로써 증명되었다.
    머신러닝에 사용되는 빅데이터는 중요하다. 데이터의 양도 중요하지만 질도 중요하다.
    데이터의 질을 높이는 방법에는 Labeling, Datenbereicherung, Datenreduktion oder Ausreißererkennung이 있다.
    이 논문에서는 이상치탐지를 통한 
    쏟아지는 빅데이터의 퀄리티를 이상치탐지를 통해 높여 더 나은 예측모델을 구축하는 것에 중점을 둔다.
    
    목적: 작업의 목적과 목표를 명확하게 기술하십시오. - 어떤 문제가 있는지

    

    독창성: 귀하의 작업이 같은 분야의 이전 연구에 어떻게 기여하고 현재 토론에 실제로 어떻게 기여하는지 강조하십시오.- 이 결과로 세상이 어떻게 나아졌는지

    작업 요약: 작업 내용에 대한 간략한 개요를 제공하고 가장 중요한 주장과 결과를 제시합니다.
    방법: 사용된 데이터, 절차 및 분석 방법을 포함하여 작업을 완료하는 데 사용한 방법을 설명하십시오.- 어떤 방법을 사용했는지

    결과: 결론 및 권장 사항을 포함하여 작업의 주요 결과에 대한 명확한 요약을 제공합니다. - 결과는 어떤지

    키워드: 작업의 주제와 사용한 방법을 설명하는 주요 키워드 목록을 포함합니다.


    Das Ziel dieser Diplomarbeit ist es, ein tieferes Verständnis für die verschiedenen Methoden zur Ausreißererkennung zu entwickeln und deren Stärken und Schwächen zu identifizieren. Durch eine gründliche Analyse der vorhandenen Technologien wird das Ziel erreicht, eine Empfehlung für die beste Methode zur Anwendung in einer bestimmten Anwendungssituation zu geben.

    Darüber hinaus soll ein Vergleich der verschiedenen Methoden auf einem realen Datensatz durchgeführt werden, um die Effizienz und Genauigkeit der Methoden zu bewerten. Die Ergebnisse dieser Studie werden wichtige Informationen für Forscher und Praktiker bereitstellen, die sich mit der Ausreißererkennung beschäftigen.

\end{abstract}
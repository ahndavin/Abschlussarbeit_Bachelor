관련 논문

주제 또는 유사한 관련 주제를 이미 다룬 사람, 설명된 솔루션 및 각 작업과 자신의 작업 사이의 연결이 무엇인지 보여줍니다.
어떤 문제가 해결되었나 어떻게 문제가 해결되었나 무엇을 얻었나 어떻게 자신의 일과 관련이 있나
이 장에서는 공통 스레드를 만들고 텍스트가 논문의 목록으로 변질되는 것을 방지하는 것이 특히 어렵습니다.

최신 논문을 우선적으로 선보이도록 합니다. 관련 분야에서 최신 연구를 계속 추적하여 최신 논문을 우선적으로 제시하는 것이 좋습니다.
선보일 논문은 일반적으로 다양한 접근방식, 실험방법, 얻은 결과 등을 제시하는 것이 좋습니다.
관련논문을 선보일 때는 주요 논문부터 선보이며, 이를 설명하는 글도 함께 작성하는 것이 좋습니다.
참조하는 논문을 정리하여 구조적으로 정리하는 것이 좋습니다.
일반적으로 관련논문 챕터는 아래와 같은 양식으로 작성됩니다.
논문 제목
저자 및 출처
요약 : 논문의 목적, 실험방법, 얻은 결과 등을 간략히 요약
관련성 : 작성하는 논문과 관련된 부분을 설명

관련 연구: 이전 연구들을 요약하여 설명합니다.
나의 연구: 연구 목적, 방법, 결과, 결론 등을 설명합니다.
결론: 이론적 배경과 관련 연구를 바탕으로 나의 연구의 의의와 제안을 설명

Verwandte Arbeiten 어떤식으로 써야하는지
	- 이상치검출이 미세먼지모델의 성능향상에 도움을 주는걸 알았다.
	- 논문에서는 iqr만을 이용하여 이상치를 검출했다.
	- 다른 분야에서 다양한 이상치검출알고리즘을 사용했다.
	- 우리 논문은 미세먼지모델을 위한 데이터세트에 iqr뿐만이 아닌
	  다른 다양한 이상치검출알고리즘을 이용하여 조사한다.
	- 또 다른 논문 한개 정도 찾아보기

Kannst du bitte einen Kapitel 'Verwandte Arbeiten' für ein Paper schreiben? Die Information sind unten:
아 두

Begründung für das Zitieren des folgendes Papers: Das Entfernen von Ausreißern hilft, die Leistung von Feinstaub-Vorhersagemodellen zu verbessern. Daher wollen wir durch die Entfernung von Ausreißern die Leistungsfähigkeit des Feinstaubvorhersagemodells unseres laufenden Projekts verbessern.
Paper: Evaluation von Ausreißer-Behandlung und Auswirkungen von Umwelteinflüssen auf Feinstaub-Analysen
Zusammenfassung: In dieser Arbeit wird untersucht, wie sich räumliche und zeitliche Ausreißer auf die Kriging-Interpolation von Feinstaubdaten auswirken. Es wurde gezeigt, dass das Entfernen von Ausreißern einen positiven Effekt hat.
Paper: Anomaly Detection and Repairing for Improving Air Quality Monitoring
Zusammenfassung: Das Papier stellt ein neuartiges KI-basiertes Framework namens AIrSense vor, um zuverlässige Schadstoffkonzentrationen aus Rohdaten zu erhalten, die von einem Netzwerk kostengünstiger Luftqualitätssensoren gesammelt wurden. Das Framework implementiert ein Verfahren zur Erkennung und Reparatur von Anomalien, um die Zuverlässigkeit von Luftqualitätserfassungsdaten zu verbessern. Experimente mit einem realen Datensatz von 12.000 Beobachtungen von 12 Sensoren, die mit realen Daten durchgeführt wurden, zeigten, dass das Entfernen von Anomalien aus Rohdaten den Fehler in den kalibrierten Werten reduzierte und die Leistung des Kalibrierungsalgorithmus verbesserte.

Begründung für das Zitieren des folgendes Papers: In den meisten Arbeiten wird nur ein Algorithmus verwendet oder nur Algorithmen, die auf einem Eigenschaft basieren, um Ausreißer im Feinstaubdatensatz zu erkennen. Also wir haben kein Paper gefunden, in dem man verschiedene Algorithmen miteinander vergleicht.
Paper: Improvement of PM Forecasting Performance by Outlier Data Removing
Zusammenfassung: Der bestehende IQR-Ansatz beurteilt Daten zu hochkonzentriertem Feinstaub als Ausreißer und entfernt diese. Dies führt zu einem Problem, da das Modell, das mit dem durch den IQR gefilterten Datensatz trainiert wurde, keine Vorhersagen für hochkonzentrierten Feinstaub machen kann. Um dieses Problem zu lösen, wurde im vorliegenden Artikel ein modifizierter IQR vorgestellt. Die Leistung von einem Modell mit Ausreißerentfernung wurde mit einem anderen Modell ohne Ausreißerentfernung verglichen, wobei eine bessere Leistung des Modells mit Ausreißerentfernung bei Vorhersagen hoher Konzentrationen in langfristigen Prognosen festgestellt wurde. Bei einigen Metriken wurde jedoch eine Abnahme der Genauigkeit beobachtet, was weitere Untersuchungen erforderlich sind.

Paper: Incremental Outlier Detection in Air Quality Data Using Statistical Methode
Zusammenfassung: Streambasierte Anomalieerkennung ist sowohl material- als auch zeitaufwändig. Um dies anzugehen, schlagen die Autoren ein Framework vor, das bei der kontinuierlichen Analyse von Datenströmen von Feinstaubsensoren eine statistische Methode zur Erkennung von Ausreißern verwendet. Die Autoren haben fünf statistische Verfahren (Z-Score, IQR, Grubb's Test, Hampel's Test, Tietjen-Moore Test) zur Erkennung von Ausreißern vergliechen und festgestellt, dass sie in der Lage sind, den Datenstrom zu analysieren. Der Leistungsunterschied zwischen der Verwendung der Methoden in einem inkrementellen Modus und den vollständigen Daten beträgt nicht mehr als 4\%. Diese Methoden können auch in kostengünstigen Analysezentren implementiert werden und erfordern keine großen Rechenressourcen, was sie für Entwicklungsländer zugänglich macht. Allerdings müssen die Methoden für höherdimensionale Daten und für hochvolumige und Hochgeschwindigkeitsdaten evaluiert werden.

Begründung für das Zitieren des folgendes Papers: Wir verweden verschiedene Algorithmen wie z. B. statistik-, cluster- und dichtebasierte Algorithmen, um die Verbesserung der Vorhersagemodellleistung durch die Ausreißerentfernung zu vergleichen.
Paper: Traffic Anomaly Detection Using K-Means Clustering
Zusammenfassung: This paper introduces Network Data Mining, the application of data mining to packet and flow data in a network, and gives a comparative overview of existing methods. A novel flow-based anomaly detection scheme based on K-means clustering is presented, where training data is divided into clusters of normal and anomalous traffic and the cluster centroids are used to detect anomalies in new monitoring data.

Paper: Local outlier factor use for the network flow anomaly detection
Zusammenfassung: The article proposes a novel approach for detecting network flow anomalies (i.e. anomalies in network behavior) in order to reduce the risk of cyberattacks. The approach is based on local outlier factor algorithm and uses aggregated network flow metrics. The article suggests 15 groups of features (74 total) for detecting anomalous network flows and the best groups of features are identified based on experimental results with the highest precision, recall, and F-measure values.

\chapterpage\chapter{Verwandte Arbeiten}
    우리는 이미 어떤 데이터세트의 이상치제거가 학습모델의 성능 향상에 큰 기여를 한다는 것을 알고 있습니다. 이는 미세먼지의 농도를 예측하는 학습모델에서도 동일하게 적용됩니다. In einem Artikel A wird untersucht, wie sich räumliche und zeitliche Ausreißer auf die Kriging-Interpolation von Feinstaubdaten auswirken. Dieser Artikel zeigt, dass das Entfernen von Ausreißern einen positiven Effekt hat. Ein anderer Artikel B präsentiert ein neues KI-basiertes Framework, das zuverlässige Schadstoffkonzentrationen aus Daten von günstigen Luftqualitätssensoren erhält. Das Framework enthält eine Anomalieerkennung und Reparatur, um die Zuverlässigkeit der Luftqualitätsdaten zu verbessern. Diese Experimente zeigen, dass das Entfernen von Anomalien die Fehler in kalibrierten Werten reduziert und die Leistung des Kalibrierungsalgorithmus verbessert.

	하지만 대부분의 작업에서는 미세먼지 데이터세트에서 이상값을 감지하기 위해 하나의 알고리즘 또는 하나의 속성에 기반한 알고리즘들만 사용됩니다. Zum Beispiel schlagt ein Artikel C einen modifizierten IQR vor, so dass es kein Problem gibt, hohe Feinstaubkonzentrationen vorherzusagen. Denn herkömmliche IQR kategorisert Daten mit hohen Feinstaubkonzentrationen als Ausreißer und eliminiert diese, was dazu führt, dass das trainierte Modell nicht in der Lage ist, Vorhersagen für diese Art von Konzentrationen zu treffen. In einem anderen Artikel D wird ein Framework vorgeschlagen, das für die Analyse von Feinstaubdatenströmen statistische Algorithmen zur Ausreißererkennung verwendet. Die hier verwendeten statistischen Algorithmen sind Z-Score, IQR, Grubb's Test, Hampel's Test, Tietjen-Moore Test. In dieser Arbeit wurden mehrere Algorithmen verwendet und ihre Leistungen miteinander verglichen, aber nur statistische Algorithmen verwendet.

	우리는 다른 분야에서 사용된 이상치탐지 알고리즘들을 이용하여 
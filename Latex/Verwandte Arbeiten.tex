관련 논문

주제 또는 유사한 관련 주제를 이미 다룬 사람, 설명된 솔루션 및 각 작업과 자신의 작업 사이의 연결이 무엇인지 보여줍니다.
어떤 문제가 해결되었나 어떻게 문제가 해결되었나 무엇을 얻었나 어떻게 자신의 일과 관련이 있나
이 장에서는 공통 스레드를 만들고 텍스트가 논문의 목록으로 변질되는 것을 방지하는 것이 특히 어렵습니다.

최신 논문을 우선적으로 선보이도록 합니다. 관련 분야에서 최신 연구를 계속 추적하여 최신 논문을 우선적으로 제시하는 것이 좋습니다.
선보일 논문은 일반적으로 다양한 접근방식, 실험방법, 얻은 결과 등을 제시하는 것이 좋습니다.
관련논문을 선보일 때는 주요 논문부터 선보이며, 이를 설명하는 글도 함께 작성하는 것이 좋습니다.
최근 5~10년간의 논문만 선별해서 사용하는 것이 좋습니다.
참조하는 논문을 정리하여 구조적으로 정리하는 것이 좋습니다.
일반적으로 관련논문 챕터는 아래와 같은 양식으로 작성됩니다.
논문 제목
저자 및 출처
요약 : 논문의 목적, 실험방법, 얻은 결과 등을 간략히 요약
관련성 : 작성하는 논문과 관련된 부분을 설명

관련 연구: 이전 연구들을 요약하여 설명합니다.
나의 연구: 연구 목적, 방법, 결과, 결론 등을 설명합니다.
결론: 이론적 배경과 관련 연구를 바탕으로 나의 연구의 의의와 제안을 설명합


\chapterpage\chapter{Verwandte Arbeiten}
    이 장에서는 동일한 주제에 대한 관련논문을 분석한다. 해당 연구에 대한 사용은 널리 연구된 문제이지만 하는 것은 연구 현장에서 비교적 새로운 주제입니다.
    
    a의 논문 A에서는 Kriging Interpolation을 사용하여 이상치를 탐지한 후 미세먼지 예측모델의 성능 변화를 분석한다.


    "Deep Anomaly Detection for Time Series Forecasting" von Wei Wei, Xiaohui Liu, Xingjian Shi, Hanjun Dai (2020)
    "LSTM-based Encoder-Decoder for Multi-Sensory Anomaly Detection" von Wenqiang Feng, Xingjian Shi, Dit-Yan Yeung (2018)
    "Time Series Anomaly Detection with Convolutional Autoencoder and Long-Short Term Memory" von Yuhan Liu, Bo Liu, Jun Liu (2019)
    "Anomaly Detection in Time Series Data Using Deep Belief Networks" von Ahmed H. Tawfik, Osama A. Mohammed (2017)
    
    "Improved Outlier Detection in Time Series Data Using Seasonality Information" von B. E. Liu et al., veröffentlicht in "IEEE Transactions on Knowledge and Data Engineering" im Jahr 2019.

    "Deep Convolutional Neural Networks for Outlier Detection in Multivariate Time Series Data" von X. Zhao et al., veröffentlicht in "IEEE Transactions on Neural Networks and Learning Systems" im Jahr 2020.

    "Anomaly Detection in Time Series Data Using Deep Learning Models" von J. Kim et al., veröffentlicht in "Expert Systems with Applications" im Jahr 2021.

    "Outlier Detection in Multivariate Time Series Data Based on Deep Auto-encoders" von Y. Gong et al., veröffentlicht in "Neural Computing and Applications" im Jahr 2020.

    "Ensemble Deep Learning for Time Series Outlier Detection" von S. N. Balakrishnan et al., veröffentlicht in "Knowledge-Based Systems" im Jahr 2022.
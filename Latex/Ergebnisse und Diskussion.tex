4. 측정 결과/분석/논의: 1/3
무엇을 하든 주석을 달고 다른 시스템과 비교하고 평가해야 합니다.
일반적으로 적절한 그래프는 접근 방식의 이점을 보여주는 데 도움이 됩니다.
주의: 각 결과/그래프에 대해 논의해야 합니다! 이 피크의 이유는 무엇이며 이 효과를 관찰한 이유는 무엇입니까?
5. 결론: 1페이지
논문에서 수행한 내용을 다시 요약하지만 이제 결과와 비교를 더 강조합니다.
발견된 결과와 논문에 제시된 토론에서 도출할 수 있는 결론을 작성합니다.
미래의 일


\chapterpage\chapter{Ergebnisse und Diskussion}
    Abschnitt \ref{sec:Evaluation}에서는 위에서 제안된 메소드를 통한 이상치의 제거가 모델의 성능을 얼마나 개선하였는지 SMAPE 메트릭을 이용하여 정량화하고 이 정량화된 결과에 대한 논의는 Abschnitt \ref{sec:Diskussion}에서 기술합니다.
    In Abschnitt \ref{sec:Evaluation} wird anhand der SMAPE-Metrik quantifiziert, wie viel die Leistung des Modells verbessert, indem die Ausreißer durch die vorgeschlagene Methode entfernt werden, und die Diskussion dieses quantifizierten Ergebnisses wird in Abschnitt \ref{sec:Diskussion} beschrieben.

    \section{Evaluation}
    \label{sec:Evaluation}
        각 메소드에 대한 모델의 성능변화를 나열하기 전에 섹션 \label{sec:Modellierung der Feinstaubkonzentration}의 Evaluation단계에서 설명되었고 평가의 기준이 되는 SMAPE에 대해 자세히 설명됩니다.
        Bevor die Leistungsänderungen des Modells für jede Methode aufgeführt werden, wird die Metrik SMAPE, die im Schritt „Evaluierung“ des Abschnitts \label{sec:Modellierung der Feinstaubkonzentration} erwähnt wird und die ein Kriterium für die Evaluierung darstellt, ausführlich erläutert.




        각 메소드를 이용한 모델의 성능변화에 대해 기술하기 전에 기존의 모델성능은 다음과 같습니다.
        
        \subsection*{IQR}
        
        \subsection*{z-Score}
        
        \subsection*{K-Means}
        
        \subsection*{LOF}
        
        \subsection*{iForest}
        
    \section{Diskussion}
    \label{sec:Diskussion}
        Diskussion
\chapterpage\chapter{Umsetzung}
    이 장에서는 실험을 위해 선택된 데이터 세트의 구조 및 데이터 전처리에 대한 프로세스에 대해 설명하고 3장에서 설명된 다섯 가지의 알고리즘의 구체적인 구현에 대해 설명한다.

    
    데이터 세트
        출처
        총 데이터 개수
        데이터 수집 기간
        센서 개수, 위치
        독립, 종속 변수 - 상관관계 - 사용되는 변수
        훈련 테스트 세트 같이 하는 이유
            - 이상치가 많기때문에 테스트세트도 제거
            - 더 많은 데이터를 이용하여 이상치를 찾기때문에 좋음
        센서를 나누는 이유(둘중 뭐가 더 좋은 결과를 내는지 보고 결정)
            Univariate Time Series Data는 멀티에서 자주 발생하는 Masking- und Swamping-Probleme을 피할수있다.
            각 시간 단위마다 여러 개의 값을 가지는 Multivariate Time Series Data는 데이터 포인트간의 밀집도가 높기에


    알고리즘 구현
        라이브러리(아다꺼랑 디플롬꺼 볼 것)

    모델 설명 짧게
        여긴 내 분야 아니니까 짧게

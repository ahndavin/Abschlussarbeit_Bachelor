\chapterpage\chapter{Umsetzung}
    이 장에서는 실험을 위해 선택된 데이터 세트의 구조 및 일명 Roh-Daten에 대한 전처리프로세스에 대해 설명하고 다음장에서는 3장에서 설명된 다섯 가지의 알고리즘의 구체적인 구현에 대해 설명한다. 마지막 장은 미세먼지 예측모델의 대략적인 구조에 대한 것이다.

    
    데이터 세트
        본 연구는 미세먼지 예측을 위해 A에서 제공되는 미세먼지 데이터세트를 사용한다. 이 기관은 x에 위치한 5개의 센서를 사용하여 해당 지역의 미세먼지농도 및 주변 기상측정값을 평균 10.3초 단위로 2020년 09월 01일부터 2022년 08월 31일까지의 기간동안 y개의 데이터를 수집했다. 이 데이터세트에는 총 네개의 features(날짜, 온도, 강수량, 바람세기)가 존재하며, 해당 독립변수와 종속변수의 단위 및 데이터타입은 표 1에 기술되어 있다. 이상치를 탐지하기 전에, 미세먼지농도와 강한 상관관계를 보이는 features들만 사용하기 위해 Pearson Correlation Coefficient (PCC)로 각각의 feature가 미세먼지농도 pm25와 어떤 관계를 갖고 있는지 분석하였다. 이는 두 변수간의 상관 관계를 계량화한 값으로, Cauchy-Schwarzsche Ungleichung에 의해 +1과 1 사이의 값을 갖는다. 일반적으로 PCC의 절댓값이 0.7 이상이면 강한 상관관계, 0.4 이상이면 뚜렷한 상관관계, 0.1 이상이면 약한 상관관계 그리고 0.1 미만이면 무시해도 좋을 상관 관계라고 해석된다. PCC로 데이터의 feature를 분석해 본 결과는 표 2와 같다.

        표1 표2
        
        따라서 피어슨 상관계수의 값이 0.4 이상인 pm25만을 최종 feature로 사용된다.

        훈련 테스트 세트 같이 하는 이유
            - 이상치가 많기때문에 테스트세트도 제거
            - 더 많은 데이터를 이용하여 이상치를 찾기때문에 좋음
        센서를 나누는 이유
            Univariate Time Series Data는 Multivariate Time Series Data에서 자주 발생하는 Masking- und Swamping-Probleme을 피할수 있다. 실제로 센서를 나누지 않은 경우 SMAPE을 기준으로 평균 1.38\% 나쁜 성능을 보였다.

    알고리즘 구현
        라이브러리(아다꺼랑 디플롬꺼 볼 것)

    모델 설명 짧게
        여긴 내 분야 아니니까 짧게
        평균 10.3초 단위로 수집된 데이터를 한 시간 단위로 평균을 내어 변환함으로써 처리해야 하는 데이터의 양을 줄이고 각 데이터의 신빙성을 높였다.
        전체 데이터세트 중 학습데이터는 2020년 09월 01일부터 2021년 08월 31일까지이며, p개의 데이터 중에 중복, 결측 및 음수의 데이터를 제외한 q개의 데이터를 사용 하였고, 평가데이터에는 2021년 09월 01일부터 2022년 08월 31일까지이며, i개의 데이터 중 위와 같은 이유로 제거된 데이터를 제외한 k개의 데이터를 이용하였다.
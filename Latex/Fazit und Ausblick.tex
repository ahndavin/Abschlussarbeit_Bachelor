작업을 간략하게 요약하고 평가하며 결과/솔루션을 결론으로 ​​요약합니다(여기서는 서론에서 제기된 문제에 대한 인사). 개발된 방법의 일반적인 유용성에 대한 평가가 제공될 수 있습니다. 팁 : 긍정적인 말로 마무리하세요! 결론은 '좋은 것'을 먼저 보여주고 '나쁜 것'을 보여줍니다. 후자의 경우 해결할 수 있으며 개발된 방법이 그럼에도 불구하고 유망하다는 점에 유의하십시오.

뭘 원햇고 그래서 뭘 햇고 어떤 결과를 얻었는지

Automatic Hyperparameter Tuning Method for Local Outlier Factor, with Applications to Anomaly Detection라는 논문은 lof에서 이웃포인트의 최적값을 찾아주는 방법에 대해 이야기한다. 해당 논문의 알고리즘을 이용하려는데 어떤 깃허브에 누군가 배포했는데 이거 쓰고 참조 하면 되는지 궁금하다.(https://github.com/vsatyakumar/automatic-local-outlier-factor-tuning)

\chapterpage\chapter{Fazit \& Ausblick}
    In dieser Arbeit wurde die fünf verschiedene Methoden zur Erkennung von Ausreißern untersucht, um den Einfluss auf die Leistung des Feinstaubprognosemodells zu analysieren. Die Ergebnisse zeigen, dass alle Methoden außer LOF die Leistung des Modells durchschnittlich um $1,75$\% verbesserten. In Bezug auf LOF scheint es jedoch eine Möglichkeit zu geben, die Leistung zu verbessern, indem der optimale Wert für die Hyperparameter bestimmt wird, wie es in \ref{Xu19} diskutiert. Laut dem Papier \ref{Soenen21} kann eine Leistungssteigerung von $13,0$\% in Form von AUC erzielt werden, wenn die optimalen Parameter eingestellt werden. Darüber hinaus können die Verwendung des in \ref{Hariri21} vorgestellten 'Extended Isolation Forest' eine bessere Leistung des Modells erzielen. Es ist jedoch zu beachten, dass die Arbeit auch einige Limitationen hat. Zum Beispiel ist der verwendete Datensatz begrenzt in Bezug auf die Anzahl und Vielfalt der verwendeten Features. In der zukünftigen Arbeit könnten größere und vielfältigere Datensätze verwendet werden, um eine bessere Vorhersageleistung zu erzielen.

    Zusammenfassend zeigt diese Arbeit, dass die Anwendung von Ausreißererkennungsmethoden bei der Vorhersage des Feinstaublevels einen positiven Effekt hat. Es gibt jedoch noch weitere Möglichkeiten, die Leistung weiter zu verbessern. In Zukunft könnten weitere Untersuchungen durchgeführt werden, um die besten Methoden für diese Art von Problemen zu bestimmen, und um die Leistung von bestehenden Methoden weiter zu optimieren.

    
    In dieser Arbeit wurde die fünf verschiedene Methoden zur Erkennung von Ausreißern untersucht, um den Einfluss auf die Leistung des Feinstaubprognosemodells zu analysieren. Die Ergebnisse zeigen, dass alle Methoden außer LOF die Leistung des Modells durchschnittlich um 1,75\% verbesserten. In Bezug auf LOF scheint es jedoch eine Möglichkeit zu geben, die Leistung zu verbessern, indem der optimale Wert für die Hyperparameter bestimmt wird, wie es in \ref{Xu19} diskutiert wurde. Laut dem Papier \ref{Soenen21} kann eine Leistungssteigerung von 13,0\% in Form von AUC erzielt werden, wenn die optimalen Parameter eingestellt werden. Darüber hinaus können die Verwendung des in \ref{Hariri21} vorgestellten 'Extended Isolation Forest' eine bessere Leistung des Modells erzielen. Darüber hinaus scheint ein Modell mit besserer Leistung implementiert werden zu können, wenn anstelle des in dieser Arbeit verwendeten Datensatzes ein Datensatz mit vielfältigeren und stärker mit der Feinstaubkonzentration korrelierenden Features verwendet wird.

    Zusammenfassend zeigt diese Arbeit, dass die Anwendung von Ausreißererkennungsmethoden bei der Vorhersage des Feinstaublevels einen positiven Effekt hat. Es gibt jedoch noch weitere Möglichkeiten, die Leistung weiter zu verbessern. In Zukunft könnten weitere Untersuchungen durchgeführt werden, um die besten Methoden für diese Art von Problemen zu bestimmen, und um die Leistung von bestehenden Methoden weiter zu optimieren.





\section{Modellierung der Feinstaubkonzentration}
해당 섹션에서는 미세먼지 예측모델을 구축하는 워크플로우에 대해 언급하고, 이 논문이 전체 머신러닝 워크플로우에서 어떤부분에 집중하는지를 그림을 통해 설명된다.

그림 1은 머신러닝의 workflow를 나타낸다. 미세먼지 예측모델도 같은 flow를 따른다.

\begin{figure}[h]
    \centering
    \includegraphics[width=\textwidth]{images/potential AI value.jpg}
    \caption{Workflow of ML}
    \label{fig:Workflow}
\end{figure}

\begin{description}
    \item[Acquisition] {머신 러닝을 하기 위해서는 기계에 학습시킬 데이터가 필요하다. 환경분야에서의 데이터 수집 프로세스는 해당 환경에 여러 물질을 스캔할 수 있는 센서들을 설치하여 수집한다. 데이터수집장치의 기술적문제나 사람의 실수 등 여러가지 이유로 수집된 데이터는 많은 누락 데이터, 매우 크거나 매우 작은 값, 정리되지 않은 데이터가 있을 수 있다.}

    \item[Inspection and exploration] {데이터가 수집되었다면, 이제 데이터를 점검하고 탐색하는 단계이다. 이 단계를 탐색적 데이터 분석(Exploratory Data Analysis, EDA) 단계라고도 하는데 이는 독립 변수, 종속 변수, 변수의 데이터 타입 등을 점검하며 데이터의 특징과 데이터 내의 여러 속성들이 의미하는 구조적 관계를 이해하는 과정을 의미한다.}

    \item[Preprocessing and Cleaning] {데이터에 대한 파악이 끝났다면, 머신 러닝 워크플로우에서 가장 까다로운 작업 중 하나인 데이터 전처리 과정에 들어간다. 이 단계는 많은 단계를 포함하고 있는데, 가령 미세먼지 예측이라면 Missing data 제거, anomaly data 제거, 데이터 cleaning, normalization 등의 단계를 포함한다.}

    \item[Modeling and Training] {데이터 전처리가 끝났다면, 머신 러닝에 대한 코드를 작성하는 단계인 모델링 단계에 들어간다. 적절한 머신 러닝 알고리즘을 선택하여 모델링이 끝났다면, 전처리가 완료 된 데이터를 머신 러닝 알고리즘을 통해 모델을 학습(training)시킨다. 모델이 데이터에 대한 학습을 마치면 이 모델는 우리가 원하는 미세먼지농도 예측을 할 수 있다.}

    \item[Evaluation] {평가단계에서는 테스트용 데이터로 모델의 성능을 평가하게 된다. 평가 방법은 모델이 예측한 데이터가 테스트용 데이터의 실제 정답과 얼마나 가까운지를 측정한다.}

    \item[Deployment] {평가를 통해 모델이 성공적으로 훈련이 된 것으로 판단된다면 완성된 모델은 배포 단계로 간다. 다만, 여기서 완성된 모델에 대한 전체적인 피드백으로 인해 모델을 업데이트 해야하는 상황이 온다면 수집 단계로 돌아갈 수 있다.}   
\end{description}

이 글에서는 이 workflow의 6가지 단계 중 Preprocessing and Cleaning 단계의 anomaly data 제거에 집중한다. 그림 1에서 보듯이 해당 단계에서 anomaly data 제거를 제거함으로써 학습데이터의 질을 향상시키고 마침내 미세먼지 예측모델은 질 좋은 데이터로 학습하게 되어 더 좋은 예측성능을 얻을 수 있다. 논문 A에서는 이상치를 통한 미세먼지예측모델의 성능을 25\% in root mean square error (RMSE) 개선했다고 한다.